\chapter{Introduction and Background}

Cities face increasing challenges from the \gls{uhi} effect. Reduced vegetation cover and extensive impervious surfaces in dense built environments creates significantly higher temperatures compared to surrounding rural areas \cite{visvanathan_mitigating_2024}. Poor climate-sensitive design can substantially impact urban livability, as people tend to avoid public spaces when conditions are too hot, cold, or windy. Effective \gls{uhi} mitigation requires comprehensive approaches that include increasing vegetation cover, implementing green infrastructure, using reflective materials, and improving building energy performance \cite{ahmed_optimizing_2024}.

The \gls{uhi} effect is influenced by the mass, orientation, heat absorption, and reflection properties of materials used in urban infrastructure. Materials like concrete and asphalt absorb and retain heat, contributing to higher temperatures. The orientation of buildings can either facilitate or obstruct airflow, affecting natural cooling. Also, materials with high reflectivity reflect more solar radiation and reduce heat absorption. In contrast, darker materials absorb more heat, exacerbating temperature increases. Dense vegetation or urban infrastructure can block natural breezes, and reduce the temperature cooling effect. Integrating vegetation with reflective, less heat-absorbent materials and optimizing building orientation can enhance the cooling benefits and improve thermal comfort in urban areas.

Modeling urban microclimate has become an important approach to understand microclimate complex interactions. \gls{cfd} tools enable data-driven urban design for designers, planners, and policymakers by supporting climate-resilient strategies and evidence-based decisions that reduce heat stress and climate-related risk through simulation \cite{schneider_pitfalls_2023, yamasaki_microclimate_2024}. The effectiveness of these simulations depends on the accuracy of the tool in representing real-world conditions and their ability to account for complex interactions between built and natural elements in urban environments \cite{su17073025}. Such tools should be able to represent both the existing and proposed urban microclimate conditions for building materials and the cooling effects of trees and greenery bodies. An urban microclimate model includes buildings and urban geometry, infrastructure (e.g. roads, paving, transit structures, plazas), shading structures and play areas, vegetation (e.g. trees, canopies, shrubs, grass, green roofs/walls), soil profiles and engineered substrates, water bodies (e.g. lakes, fountains, rivers, drainage systems), and surface material properties (e.g. albedo, roughness, permeability, thermal capacity).

High vegetation density produces a blocking effect, lowering the effectiveness on thermal comfort improvement in street canyon settings \cite{kianmehr_quantifying_2022}. Kianmehr found that dense vegetation did not decrease the \gls{utci} level in shallow canyon settings. However, increasing canyon depth led to a reduction in \gls{utci}, indicating a deterioration of thermal comfort even under similarly dense vegetation conditions. Huo and Chen \cite{huo_study_2024} categorized vegetation cooling effects into direct cooling effects and indirect cooling effects. Direct cooling effects include the plant evapotranspiration, and the solar radiation absorption due to photosynthesis and indirect cooling effects include leaf shielding and reflection, and heat transfer enabled by the temperature difference between the area under trees and the surrounding areas.

\clearpage
The cooling magnitude of the air temperature varies depending on the location settings and the climate conditions, and it is not linearly related to tree coverage \cite{tan_evaluating_2022}. Other researchers reported air temperature values decreasing from 'not significant' up to 2ºC:

\begin{itemize}
    \item Decrease of 0.7 – 1.5 in Gothenburg \cite{konarska_transpiration_2016}
    \item Decrease of 0.5 - 1.0 in Negev Highlands \cite{shashua-bar_cooling_2009}
    \item Decrease of 2 in Phoenix, Arizona \cite{middel_impact_2016}
    \item Marginal decrease in Manchester, UK \cite{armson_comparison_2013}
\end{itemize}

\section{Microclimate simulation tools}
%introduction, mention other tools, cfd, etc.
%add comparative table of tools
Microclimate simulation tools integrate complex physical processes, including short-wave radiation (direct solar radiation that heats surfaces and materials), long-wave radiation (thermal energy re-emitted by warm surfaces), wind flow, vegetation processes, and surface energy balance to simulate air temperature, humidity, and thermal comfort indices. Multiple urban microclimate modeling approaches have been developed over the past two decades, ranging from simple radiation-focused models to comprehensive coupled three-dimensional systems. The principal microclimate simulation tools available to researchers and practitioners present particular strengths and limitations that influence their applicability, accuracy, and computational requirements, as summarized in \ref{tab:microclimate_tools_comparison}.

\clearpage
\begin{table}[h!]
    \centering
    \caption{Comparison of microclimate and urban climate modeling tools.}
    \label{tab:microclimate_tools_comparison}
    \small
    \begin{tabularx}{\textwidth}{l X}
        \toprule
        \textbf{Model Feature} & \textbf{Details} \\
        \midrule
        
        % --- RayMan ---
        \multicolumn{2}{@{}l}{\textbf{RayMan}} \\
        \midrule
        Model Type & Diagnostic radiation model \\
        Spatial Resolution & Point-based \\
        Output Variables & Tmrt, thermal comfort indices \\
        Key Strengths & Simple, fast, user-friendly \\
        Main Limitation & Single-point calculations only \\
        \midrule
        
        % --- SOLWEIG ---
        \multicolumn{2}{@{}l}{\textbf{SOLWEIG}} \\
        \midrule
        Model Type & Raster-based radiation model \\
        Spatial Resolution & 1--5 m grid \\
        Output Variables & Tmrt, radiation fluxes \\
        Key Strengths & Spatial resolution, GIS integration \\
        Main Limitation & Simplified diffuse radiation treatment, rasterization artifacts \\
        \midrule
        
        % --- UMEP ---
        \multicolumn{2}{@{}l}{\textbf{UMEP}} \\
        \midrule
        Model Type & Coupled multi-scale system \\
        Spatial Resolution & 1--10 m (scale-independent) \\
        Output Variables & Multiple meteorological variables \\
        Key Strengths & Modular, open-source, thermal comfort \\
        Main Limitation & Limited 3D dynamics, optimized for 100 m+ scale \\
        \midrule
        
        % --- STEVE ---
        \multicolumn{2}{@{}l}{\textbf{STEVE}} \\
        \midrule
        Model Type & Simplified screening tool \\
        Spatial Resolution & Buffer zones (50 m) \\
        Output Variables & Air temperature \\
        Key Strengths & Rapid assessment, tropical calibrated \\
        Main Limitation & Low resolution, calm conditions only \\
        \midrule
        
        % --- SOLENE-microclimat ---
        \multicolumn{2}{@{}l}{\textbf{SOLENE-microclimat}} \\
        \midrule
        Model Type & Coupled CFD--thermal system \\
        Spatial Resolution & 0.5--2 m (CFD-dependent) \\
        Output Variables & Wind, temperature, building energy \\
        Key Strengths & Complex geometry, coupled physics \\
        Main Limitation & Requires expertise, high computational demand, no GUI \\
        \midrule
        
        % --- ENVI-met ---
        \multicolumn{2}{@{}l}{\textbf{ENVI-met}} \\
        \midrule
        Model Type & 3D non-hydrostatic model \\
        Spatial Resolution & 0.5--10 m \\
        Output Variables & Comprehensive: radiation, wind, temperature, vegetation, energy, comfort \\
        Key Strengths & Holistic coupling, 3D vegetation, detailed building physics, proven validation \\
        Main Limitation & Computational demand, learning curve \\
        \bottomrule
    \end{tabularx}
\end{table}

\clearpage
\subsection{RayMan}
%why are they different from what im doing, why it is not the best
RayMan is a well-established diagnostic tool for computing \gls{mrt} and thermal comfort indices (e.g. PET, UTCI) based on simple meteorological inputs including global radiation, air temperature, wind speed, and varying clothing level that a person will wear outdoors. The model has gained widespread adoption due to its accessibility and rapid computational speed, calculating \gls{mrt} at a given location. However, RayMan presents restrictive limitations for comprehensive microclimate analysis in complex urban environments, as shown in \ref{tab:microclimate_tools_comparison}.

The main limitation of RayMan is that it simulates microclimate conditions only at selected point locations, calculating \gls{mrt} based on user-provided solar altitude and azimuth. Point-based results capture conditions at discrete locations, but fail to represent the continuous spatial variability of surface temperatures and radiation patterns \cite{Matzarakis2007}. Surface-based mapping, in contrast, enables gradient maps that reveal spatial variations in thermal comfort and mean radiant temperature (\gls{mrt}); without such data, localized heat stress can be underestimated. RayMan also produces errors at low solar altitudes, as in winter or high-latitude sites \cite{NYGARDRIISE2024112975}, and cannot account for reflected shortwave radiation from surrounding surfaces, an important heat source in dense urban areas. Advanced urban microclimate studies therefore require site-specific, surface-based approaches to accurately assess thermal environments and support evidence-based urban design.

\subsection{SOLWEIG}
%why are they different from what im doing
SOLWEIG (Solar and Longwave Environmental Irradiance Geometry) is a raster-based radiation model designed to simulate spatial variations of \gls{mrt} and radiation in detailed urban microclimate settings \cite{solweig}. The SOLWEIG model calculates shortwave and longwave radiation fluxes and derives \gls{mrt} using angular and view factors, which account for shading, reflections, and the fraction of sky and surrounding surfaces “seen” from a point, enabling more accurate thermal comfort estimates. The model's primary limitation is its simplified treatment of diffuse and reflected solar radiation, which is particularly important in dense urban settings where indirect radiation from buildings, pavements, and other surfaces dominates the thermal environment. A second limitation is that SOLWEIG sacrifices three-dimensional geometric detail by requiring pixelated representations of buildings and topography. These pixelated representations tend to smooth away features such as narrow alleys, building overhangs, and recessed entrances that influence local microclimate by altering wind patterns, shading geometry, and thermal mass interactions \cite{lindberg2011influence}. SOLWEIG does not simulate wind flow or dynamic surface energy balance processes beyond radiation, which are critical for capturing convective and evaporative cooling, heat storage in surfaces, and localized variations in thermal comfort.

\subsection{STEVE}
STEVE (Screening Tool for Estate Environment Evaluation) is an ArcGIS plugin that predicts daily minimum, average, and maximum air temperatures \cite{steve}. Its integration with ArcGIS allows urban planners, geographers, and environmental engineers to incorporate microclimate analysis into spatial mapping. The minimum temperature helps identify cold stress or nighttime cooling, the average temperature provides an overall daily thermal assessment, and the maximum temperature highlights potential heat stress periods, supporting evaluation of urban design strategies for thermal comfort. To predict daily air temperature, the model relies on empirical data for a 3-year period, focusing on the estate development scale. This empirical training data set is more appropriate for tropical climate zones, meaning that the tool can be more accurate for specific locations compared to other climate zones \cite{steve_envimet}. While STEVE can accurately predict air temperature values, it only assumes calm wind conditions, which is important because wind affects convective cooling, thermal comfort, and microclimate variability; neglecting airflow dynamics can lead to overestimation of heat stress in areas with significant ventilation. Also, STEVE predicts air temperature at specific points of interest, requiring mapping within GIS, which is limiting because it does not capture continuous spatial variability in urban environments. On the contrary, complete microclimate tools like ENVI-met store data at each voxel centroid, giving designers high-resolution spatial information to identify localized heat or cool areas and optimize urban layouts for thermal comfort.

\subsection{SOLENE-microclimat}
SOLENE-microclimat uses a three-dimensional numerical model to simulate building thermal behavior and exterior airflow dynamics \cite{MORILLE20151165}. It couples the CFD tool Code-Saturne with the thermo-radiative tool SOLENE, integrating a radiative model, a thermo-radiative model, CFD, and a Building Energy Simulation (BES) component. This combination allows researchers to evaluate the interactions between building energy performance, airflow, and outdoor thermal comfort at high spatial and temporal resolution, addressing complex urban microclimate problems that simpler models cannot capture. The tool is primarily aimed at specialized researchers because its setup, calibration, and interpretation require advanced knowledge of CFD, radiation modeling, and building physics, limiting accessibility for non-specialist users. For example, SOLENE-microclimat uses the \gls{lai} as input, including a soil profile and water bodies. To implement these parameters, users need to manipulate its parameters using a Python environment. Using Python enables more customization, but also requires higher user expertise for implementation \cite{imbert2018simulation}. Despite its implementation complexity, many SOLENE-microclimat modules are validated by researchers, including the radiative model and the greenery models. However, other modules are not validated yet due to the wide number of parameters and specific locations referencing empirical data \cite{BOUZOUIDJA2021107556}.

\subsection{ENVI-met}
ENVI-met is a three-dimensional microclimate simulation software developed by Bruse and Fleer \cite{bruse_simulating_1998}. This software cover scientific disciplines from fluid dynamics and thermodynamics, to plant physiology and soil science \hyperlink{https://envi-met.com/microclimate-simulation-software/}{(envi-met.com)}. The main features of ENVI-met included solar analysis, pollutant dispersion, building physics, green and blue technologies, wind flow, outdoor thermal comfort, vegetation analysis, and humidity. It allows the user to assess solar radiation exposure for building facades, including vegetation environmental factors. Additionally, ENVI-met offers detailed environmental analysis, including trees and wind flow patterns.

ENVI-met uses an orthogonal Arakawa C-grid for the simulation domain 3D model and the Finite Difference Method to solve the partial differential equations \cite{salata_urban_2016}. The Finite Difference Method enables ENVI-met to efficiently solve the complex equations governing airflow, heat transfer, and radiation, supporting high-resolution simulation of urban microclimate processes. By using a fully implicit scheme, ENVI-met can take larger simulation time steps without producing unstable or unrealistic results. In other words, this approach allows the model to calculate changes over longer periods while keeping the numerical solution stable and reliable. However, ENVI-met presents the limitation of only updating air temperature and relative humidity boundary conditions during the simulation process \cite{acero_comparison_2015}. Researches have focused on testing and validating ENVI-met simulation results against real-world measured data. Regardless of inherent microclimate model's limitations, ENVI-met is considered the adequate state-of-the-art tool for studying micro-climatic conditions \cite{chow_observing_2011, chow_assessing_2012, emmanuel_urban_2007}.

ENVI-met is considered a well-established microclimate simulation software, used by many researchers aiming to compare and validate the accuracy of simulation results. Studies consists of comparison between simulation data to micro-meteorological station data, comparing ENVI-met results to observed conditions in urban settings. Simulation results hardly are a perfect match due to each models' limitations. One of the most relevant techniques to measure simulation results validity against real-world data is the Willmott's Index of Agreement (d). Using the Willmott's Index of Agreement over correlation-based performance is preferred as correlation often shows good predictors versus the latter when they are not \cite{petri_planning_2019}.

Several studies have found relevant insights into the simulation results given by ENVI-met. These simulation results accuracy may vary depending on the forcing conditions and meteorological conditions used for simulation \cite{salvati_microclimate_nodate}. Related studies found either overestimation or underestimation of measurements. For example, Le and Chan \cite{le_comparison_2023} state that applying greenery overestimates its cooling effect. Adelia et al. \cite{adelia_tool_2020} compares ENVI-met wind flow with ANSYS Fluent results, finding that ENVI-met does not fully capture turbulence flows generated by building elements. Studies comparing ENVI-met against air temperature measurements show a high value of d=0.8 \cite{koletsis_validation_nodate}. However, the authors report high \gls{rmse} and \gls{mae} values for relative humidity, indicating possible errors in the ENVI-met simulation process. They also obtained a low agreement value of d=0.5 for wind speed, arguing that ENVI-met cannot incorporate real-world wind speed and direction.

Researches found that ENVI-met models tend to overestimate the daytime shortwave radiation values \cite{acero_comparison_2015, huttner_further_2012, salata_urban_2016, sharmin_microclimatic_2017}. Also, the assumption of a static sky condition and steady wind profile results in excluding transient clouds. Considering constant cloud cover affects the deviation between modeled and experimentally calculated \gls{mrt} \cite{acero_comparison_2015, morakinyo_thermal_2019, salata_urban_2016}. Assuming a static sky and steady wind excludes transient cloud effects, which can cause ENVI-met to over- or underestimate \gls{mrt} on partly cloudy days, reducing accuracy in capturing short-term variations in thermal comfort. For airflow dynamics, ENVI-met is rarely evaluated for its ability to reproduce wind speed profiles. Given an initial wind speed direction and magnitude, it remains almost constant during the simulation process. As a result, ENVI-met tends to overestimate wind speed values, with higher differences for wind speed initialized above 2 m/s \cite{acero_evaluating_2018, kruger_impact_2011}.

\section{Problem statement}
Urban areas are increasingly affected by the \gls{uhi} effect, driven by rapid urbanization, climate change, and more frequent heatwaves. Vegetation can provide cooling, but its effects are highly context-dependent, making it difficult to quantify. Predicting microclimate conditions at the neighborhood and district scale is challenging because Computational Fluid Dynamics (\gls{cfd}) tools must balance accuracy, complexity, and practicality. \gls{cfd} tools vary in terms of capabilities, ease of implementation, and context model representation. When modeling urban microclimate settings, researchers and practitioners need to use custom \gls{cfd}, radiation, and heat transfer models and platforms to fully capture all urban microclimate interactions. To model these interactions, computational tools provide specific functionality such as exposing parameters, user interfaces, post-processing utilities, and in some cases custom scripting capabilities. This set of functionalities is much more utilized by professional practice settings. Hence, it is crucial to test, calibrate, and ultimately validate microclimate simulation tools that aim for accuracy, seamless operation and implementation, and substantially lower computational cost.

\section{Research hypothesis}
This thesis focuses on designing and testing a urban microclimate modeling workflow to compare raw microclimate indicators: air temperature, specific humidity, and wind speed. The main goal of microclimate simulation tools is to appropriately and accurately represent real-life scenarios. However, these tools tend to differ in the accuracy of simulation results due to assumptions, boundary conditions, simplifications, and limitations in either the geometry representation or the solving models. Ideally, simulation results should be compared against real measurements relying on highly accurate weather stations or high-resolution sensors. There is a significant challenge when there is a lack of real measurements, requiring the use of other well-established microclimate simulation tools to establish a reliable benchmark. The work builds the case for a comparative study between ENVI-met and the urbanMicroclimateFoam solver operated and modeled by the Grasshopper plugin Outdoor+.

Existing microclimate simulation tools have limitations that hinder reliable \gls{uhi} and vegetation cooling predictions. ENVI-met, while widely used and validated, shows systematic biases in wind speed, radiation, vegetation cooling estimation, and under-represents turbulence and transient meteorological variations. This study uses ENVI-met to explore, compare, and ultimately calibrate the Outdoor+ plugin that handles the urbanMicroclimateFoam solver. The urbanMicroclimateFoam solver can model detailed urban geometry and multiple physical processes, such as airflow, heat transfer, and radiation. However, it does not include an easy-to-use interface, so users must manually prepare input files and analyze results, often using separate software tools. For example, a researcher might need to generate building and street geometries in a CAD program and then post-process airflow data using a visualization tool. Since direct comparisons with real-world measurements are often infeasible, due to the complexity of urban environments, limited availability of high-resolution observational data, and the cost of extensive field campaigns, there is a need to benchmark Outdoor+, the Grasshopper plugin that operates the urbanMicroclimateFoam solver, against established tools like ENVI-met.

The hypothesis of this comparative study between Outdoor+ and ENVI-met focuses on the reliability of simulation results and the implementation in professional practice. To successfully implement Outdoor+ in professional practice, non-expert users can rely on the Outdoor+ user interface to set up an urbanMicroclimateFoam case; however, uncertainties remain due to user-defined input parameters, underlying model assumptions, and the need for proper validation to ensure reliable results. Creating an urbanMicroclimateFoam case using a design-oriented platform used in design environments allows them to adopt a more accurate, flexible, and computationally cheaper tool. Providing a validated parametric modeling tool for design practitioners allows urban microclimate modeling to be embedded within a typical early design stage process to inform massing, orientation, material selection, and vegetation strategies.

\section{Research aim and objectives}
This comparative study aims to provide professional practitioners results for microclimate simulation validity and feasibility in professional practice. To achieve this validity and feasibility, the following objectives are necessary:

\begin{itemize}
    \item Developing a specialized comparative workflow between ENVI-met and Outdoor+, matching meteorological settings, the buildings material model, and the vegetation model.
    \item Performing a statistical analysis to evaluate the overall agreement between two microclimate simulation datasets.
    \item Exposing visual distribution patterns to evaluate the model's response to the proposed case study geometries.
    \item Communicating implementation feasibility for fast-paced early-design stage processes.
\end{itemize}

\clearpage
\section{Thesis structure}
The structure of this thesis is organized into five chapters. Chapter 1 introduces the research background for urban microclimate conditions and simulation approaches, including the general and specific problem, and evaluation of other microclimate simulation tools. Chapter 2 presents the methodology, describing the workflow for the two case studies: stadium typology and educational center typology. The methodology section describes meteorological inputs, geometry pre-processing, building material and vegetation model matching, simulation settings, data collection procedures, and statistical analysis for agreement between ENVI-met and Outdoor+. Chapter 3 describes the simulation and statistical results of the two case studies, consisting of statistical metrics tables, heatmap figures, scatter plots, and timeline plots for air temperature, specific humidity, and wind speed. Chapter 4 discusses the implications of the agreement and differences in air temperature, specific humidity, and wind speed, testing Outdoor+ performance as "predicted" data against ENVI-met "observed" data. Chapter 5 concludes the thesis with a summary of key contributions and recommendations for future work.


