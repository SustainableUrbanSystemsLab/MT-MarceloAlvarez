\chapter{Discussion}
%background imformation: research purposes, theory and methodology
This thesis presented a comparative study between Outdoor+ and ENVI-met, testing for raw microclimate indicators such as air temperature, specific humidity, and wind speed. On one hand, this study treats ENVI-met as the \textit{source-of-truth} microclimate simulation tool, obtaining simulation results for the stadium and educational center case studies. On the other hand, we use Outdoor+ as the interface to write, modify, and execute the urbanMicroclimateFoam solver process. The statistical comparison of both tools is done by matching both models' meteorological settings, building material properties, and vegetation model parameters. The main methodological goal is to isolate both tools' solving capabilities by matching its settings with a focus in practical implementation design environments. In this section, the stadium typology and educational center typology case studies results are reported. Key results include statistical metrics performance, heatmap patterns, scatter plots explaining each dataset, and timeline plots exposing the air temperature, specific humidity, and wind speed values over time for both tools.

\section{Case studies}
%summarizing and reporting key results: explain what the results mean
The comparative analysis of the stadium and educational center case studies reveals consistent and systematic differences between Outdoor+ urbanMicroclimateFoam and ENVI-met tools, highlighting how solver design, boundary condition handling, and model resolution influence simulation outcomes. Statistical metrics for both cases show poor-to-moderate overall agreement, with air temperature performing best relative to specific humidity and wind speed. For the stadium case study, the moderate air temperature \gls{rmse} of 2.30°C and d value of 0.65 indicate that both tools capture general temperature trends reasonably well. However, ENVI-met tends to slightly overestimate values. By contrast, the educational center case study exhibits a substantially higher air temperature \gls{rmse} of 4.63°C and a notably low d value of 0.19, reflecting a divergence in tool simulation results under more complex building and vegetation configurations. The negative $R^2$ values for all variables in both case studies reveals a more limited explanatory capability of ENVI-met results when compared to Outdoor+ simulation results.

Specific humidity and wind speed values maintain poor agreement for both case studies. In the educational center case study, the specific humidity d value of 0.33 and wind speed d value of 0.40 are consistent with the low agreement observed in the stadium case, showing that ENVI-met struggles to capture detailed microclimate interactions. These findings confirm that the two tools operate substantially differently when simulating raw environmental factors. This operation difference happens particularly under conditions where vegetation effects, building shading, and fine-scale aerodynamic phenomena become influential.

Heatmap analyses expose meaningful spatial and temporal patterns that persist across both case studies. For the stadium case study, Outdoor+ captures detailed variations influenced by buildings and vegetation, while ENVI-met produces more uniform distributions driven by simplified boundary condition treatments. This contrast becomes even more pronounced in the educational center case study. ENVI-met again shows uniformly distributed specific humidity values at each time step, with only subtle morning and afternoon differences. On the other hand, Outdoor+ reveals a broader range of values dependent on time of day and spatial context. Outdoor+ consistently captures lower specific humidity in the morning hours and gradually higher values in the afternoon, whereas ENVI-met trends in the opposite direction during early hours and fails to represent localized evapotranspiration effects. This divergence confirms that solver physics and geometry processing critically influence the ability to simulate fine-scale moisture dynamics in the studied urban layouts.

Wind speed heatmaps and distributions show similar tool-dependent patterns. For both case studies, Outdoor+ urbanMicroclimateFoam captures pronounced spatial variations due to building edges, corner accelerations, and vegetation drag, while ENVI-met maintains a steady, nearly uniform wind flow with limited variation over time. In the educational center case study, Outdoor+ predicts wind speeds that vary widely, while ENVI-met remains confined to values below 2.5 m/s with negligible temporal dynamics. The consistency of these patterns across two different case study geometries reinforces the robustness of Outdoor+ in capturing domain-specific aerodynamic behavior.

Scatter plots and timeline plots provide additional evidence supporting these trends. For both case studies, Outdoor+ aligns more closely with boundary reference values for air temperature during peak hours, while ENVI-met tends to generate lower morning values followed by a midday rise. In the educational center case study, Outdoor+ air temperature begins significantly higher in the morning, gradually decreasing after peak hours, whereas ENVI-met follows an opposite trajectory. Specific humidity timeline results again demonstrate Outdoor+ alignment with reference values, while ENVI-met produces morning underestimations and midday peaks inconsistent with boundary conditions. Wind speed timelines maintain the same contrast across both case studies: ENVI-met produces horizontal, steady curves, while Outdoor+ captures the expected temporal variation from morning to afternoon.

Overall, the combined findings for the stadium and educational center case studies indicate that while both tools reproduce general air temperature trends to some degree, Outdoor+ urbanMicroclimateFoam consistently demonstrates a higher capacity to capture fine-scale interactions essential for urban microclimate modeling. The consistent patterns across two different typologies highlight the importance of solver physics models, mesh refinement, boundary condition implementation, and geometry processing when selecting simulation tools for advanced environmental design applications. ENVI-met simplifications may be acceptable for broad temperature assessments but remain limited in their ability to represent detailed humidity and wind effects. Together, these case studies demonstrate the practical implications for tool selection, reaffirming that Outdoor+ provides a more reliable and detailed framework for capturing urban microclimate variability and guiding informed design decisions.

\section{Comparative study findings}
%comment on key results: why the results matter
%identify patterns and relationships
%results met or didn't met expectations/did or didn't support hypothesis
%explain unexpected findings and their importance: do these findings comfirm or challenge current theories? what practical implications are there, if any?
This comparative study shows that Outdoor+ urbanMicroclimateFoam captures more detailed interactions in typical professional practice project typologies than ENVI-met. Both tools perform similar for predicting air temperature, aligning with measured boundary reference trends. However, ENVI-met shows simplifications in simulating specific humidity and wind speed. These simplifications are consistent with literature reporting ENVI-met limitations in modeling fluid dynamics. The observed substantial differences in specific humidity and wind speed were unexpected and confirm that Outdoor+ and ENVI-met operate differently, due to differences in governing equations and geometry processing.

The detailed patterns captured by Outdoor+ highlight the importance of precise geometry representation for microclimate simulations. A precise geometry representation refers to both the initial input 3D geometry, and the internal geometry pre-processing to generate a simulation-ready geometry. Despite the simulation-ready geometry pre-process, Outdoor+ demonstrates significantly lower computational times, making it suitable for early design stages. However, the tool requires greater user expertise for 3D modeling, whereas ENVI-met mitigates input geometry errors through fixed cell size override. These results suggest that while Outdoor+ is viable for professional adoption, appropriate training is necessary to seamlessly embed in early design stages.

\section{Limitations and future work}
%limitations of the study:
%this analysis has concentrated on...
%the findings of this study are restricted to...
%this study has addressed only the question of...
Urban microclimate settings are diverse in terms of meteorology and typology. In the universe of microclimate simulation, there are several tools that aim to represent and predict urban microclimate phenomena. This comparative study has concentrated on comparing Outdoor+ urbanMicroclimateFoam with the state-of-the-art ENVI-met. Specifically, comparing the overall agreement between simulated values for air temperature, specific humidity, and wind speed. These three raw environmental factors are used as inputs for computing human-oriented thermal comfort metrics, such as UTCI and PET. Therefore, the findings of this study are restricted to the base microclimate simulation results without calculating thermal comfort metrics. Outdoor+ version 0.0.19-beta (used here) does not include functionality to internally calculate these thermal comfort metrics. However, working with raw environmental factors as proposed in this comparative study do not limit its usability in both research and professional practice. Particularly, professional practice can benefit from understanding more detailed urban microclimate patterns for custom urban geometry setup.

%make recommendations for future implementations and/or future research: what's next
Future research should focus on modeling a wider range of typical urban typologies with different climate zones. More urban typology variations with their respective climates would further expose the similarities and differences between Outdoor+ and ENVI-met. As discussed, the 3D geometry model representation is critical for capturing more detailed interactions between the air, buildings, and vegetation elements. Theoretically, ENVI-met is capable of simulating geometries with finer cell sizes, however, the computational cost is extremely high and simulation times are often not feasible for design environments. On the other hand, Outdoor+ allows a better geometry representation with finer cells while maintaining reasonable computational cost. However, the solver is very sensitive to the quality of input geometry.

This comparative study methodology can be further improved to include outdoor thermal comfort metrics. While raw environmental factors such as air temperature, specific humidity, and wind speed are core components for describing urban microclimate phenomena, established thermal comfort metrics are more relevant for human-centered design outcomes. Future versions of Outdoor+ will include outdoor thermal comfort metric utilities to compute such indicators.

Finally, ENVI-met is well-known as a comprehensive and complete microclimate simulation tool that includes more entities than Outdoor+ version 0.0.19-beta. Entities such as soil profiles, custom 3D terrain, and their respective accurate boundary condition settings, limit the model matching. Outdoor+ with soil and terrain modeling capabilities should capture even more detailed interactions within the simulation domain, allowing to expose new patterns or determine how sensitive an urban microclimate model is to those entities.
