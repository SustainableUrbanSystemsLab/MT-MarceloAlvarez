\chapter{Conclusion}

This thesis presented a comparative study between Outdoor+ and ENVI-met, focusing on raw microclimate indicators: air temperature, specific humidity and wind speed. By harmonizing the simulation settings, we isolated both tools' solver performance, modeling the fine-scale interactions in complex urban environments. The practical and methodological contribution of this study lies in designing and executing a robust and advanced microclimate simulation modeling workflow, from geometry generation to simulation results comparison.

This research brings key contributions to the field of urban microclimate modeling by providing a structured and replicable comparative methodology between a widely established state-of-the-art tool and a more advanced open-source CFD-based solver. The study highlights systematic differences in how each tool represents airflow, heat, and moisture transport at the fine scale, emphasizing the role of solver physics, boundary condition formulation, and geometry sensitivity in shaping simulation outcomes. By testing both tools under harmonized conditions and across two distinct urban typologies, this research strengthens the understanding of tool-specific behavior and reliability in design-driven applications.

This work also fits within the context of past research by extending existing validation and comparison efforts of ENVI-met toward a newer generation of coupled multi-physics solvers such as urbanMicroclimateFoam. While previous studies have primarily focused on validation against field measurements or tool-specific performance, this thesis contributes a direct tool-to-tool comparison under controlled modeling conditions. In doing so, it reinforces previously reported limitations of ENVI-met related to wind speed and humidity dynamics, while also demonstrating the potential of open-source CFD-based approaches for more detailed microclimate representation.

The significance of the results lies in the consistent trends observed across both case studies. Outdoor+ demonstrated greater sensitivity to geometry, shading, and boundary conditions, as well as a wider variability in air temperature, specific humidity, and wind speed. In contrast, ENVI-met showed more uniform distributions, particularly for wind speed and humidity. These findings are important for designers and researchers because they directly affect the interpretation of microclimate performance and thermal comfort assessments. The results confirm that tool selection can strongly influence design decisions when fine-scale microclimate behavior is critical.

This study also presents an additional way of thinking about the urban microclimate simulation workflow by shifting focus from solely using closed, pre-defined simulation environments toward more flexible, parametric, and CFD-based modeling platforms. The integration of Outdoor+ within a parametric design environment introduces new opportunities for iterative design exploration, sensitivity testing, and geometry driven performance evaluation. This opens future directions for combining high-fidelity microclimate solvers with generative and performance driven urban design workflows.

Despite its contributions, this study has limitations related to the number of case studies, the exclusion of soil and terrain modeling, and the focus on raw microclimate indicators rather than human-centered thermal comfort metrics. These constraints limit the generalization of the findings across urban typologies and climates. Future work should expand the scope to additional urban configurations, diverse climate zones, and the inclusion of thermal comfort indices such as MRT, PET, and UTCI.

Overall, this thesis demonstrates that Outdoor+ is a feasible and flexible alternative for advanced urban microclimate analysis when fine-scale resolution, parametric control, and solver transparency are required. Regardless of the platform used, this work confirms that urban microclimate modeling must balance physical accuracy, computational efficiency, and usability to effectively support climate-responsive urban design.
